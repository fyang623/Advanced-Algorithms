\documentclass{article}[11pt]

\addtolength{\oddsidemargin}{-.875in}
\addtolength{\evensidemargin}{-.875in}
\addtolength{\textwidth}{1.77in}
\addtolength{\textheight}{0.3in}

\begin{document}

\newcommand{\E}{\textbf{E}}
\newcommand{\var}{\textbf{var}}

\begin{center}
CMSC 651, Spring 2018, University of Maryland \\
Project, due as PDF to the address \emph{cmsc651.umd@gmail.com} by 11:59PM on May 10, 2018
\end{center}

\medskip \noindent
\textbf{Notes:} (i) Please work on this with your group (one writeup per group). Consulting
other sources (including the Web) is not allowed. 
(ii) Write your solutions neatly and \emph{include your names}; if you are able to make partial progress by making some
additional assumptions, then \textbf{state these assumptions clearly and submit
your partial solution}. 
(iii) \textbf{Please make the subject line of your email} ``\textit{651 Project}" followed by
your full name, and email a PDF to \emph{cmsc651.umd@gmail.com}: PDF generated from Word or LaTeX strongly encouraged. 



\medskip \smallskip \noindent
0. Read Chapter 5 of the Blum-Hopcroft-Kannan book (``BHK'' below) posted in Resources up to, and including, the end of Section 5.14. \emph{This is a fair amount of material, so start early! It is also fun.} 

\medskip \noindent
1.  Exercise 5.2 from BHK. ~~\textbf{(10 points)} 


\medskip \noindent
2.  Exercise 5.3 from BHK. ~~\textbf{(10 points)} 


\medskip \noindent
3.  Exercise 5.6 from BHK. ~~\textbf{(10 + 10 + 5 points)} 


\medskip \noindent
4.  Exercise 5.7 from BHK. ~~\textbf{(15 points)} 

\medskip \noindent
5.  Exercise 5.8 from BHK. ~~\textbf{(10 + 10 + 10 points)} 


\end{document}
