\documentclass{article}[11pt]

\addtolength{\oddsidemargin}{-.875in}
\addtolength{\evensidemargin}{-.875in}
\addtolength{\textwidth}{1.77in}
\addtolength{\textheight}{0.3in}

\begin{document}

\newcommand{\E}{\textbf{E}}
\newcommand{\var}{\textbf{var}}

\begin{center}
CMSC 651, Spring 2018, University of Maryland \\
HW4, due as PDF to the address \emph{cmsc651.umd@gmail.com} by 11:59PM on April 18, 2018
\end{center}

\medskip \noindent
\textbf{Notes:} (i) Please work on this with your group (one writeup per group). Consulting
other sources (including the Web) is not allowed. 
(ii) Write your solutions neatly and \emph{include your names}; if you are able to make partial progress by making some
additional assumptions, then \textbf{state these assumptions clearly and submit
your partial solution}. 
(iii) \textbf{Please make the subject line of your email} ``\textit{651 HW4}" followed by
your full name, and email a PDF to \emph{cmsc651.umd@gmail.com}: PDF generated from Word or LaTeX strongly encouraged. 

 
\medskip \noindent
1. Complete the quick proof we showed in class that the sum of the elements of an $n$-element array can be computed using $n$ processors and in $O(\log n)$ time on an EREW PRAM. ~~\textbf{(12 points)}

\medskip \smallskip \noindent
2. In studying how to do processor emulation with fewer processors, we considered the following algorithm $A$ on a DAG $G = (V,E)$:

\begin{verbatim}
i = 0;
while (G is not empty) {
   i = i + 1;
   L(i) = current set of sources (nodes with no incoming edges) in G;
   Remove L(i) (and all edges incident on vertices in L(i)) from G;
}
\end{verbatim}
Suppose $A$ partitions $V$ into $L(1), L(2), \ldots, L(\ell)$. We used the following fact without proof, in proving that our processor assignment led to a $2$-approximation of the maximum completion time: \emph{$\ell$ equals the number of vertices in a longest (directed) path in $G$}. Prove this fact.
 ~~\textbf{(15 points)}

\medskip \noindent
3. Problem 5.3 from the Lau-Ravi-Singh book that is posted under ``Resources''.
~~\textbf{(10 + 8 points)}

\medskip \noindent
4. Problem 5.6 from the Lau-Ravi-Singh book that is posted under ``Resources''. (The notation ``$X - Y$ paths'' does not mean ``$X$ minus $Y$'', but refers to ``from $X$ to $Y$''.) 
~~\textbf{(15 points)}

\end{document}